\thesistitle{Spin torque driven magnetization dynamics in nanoscale magnetic tunnel junctions}

%"Dissertation" for PhD, "Thesis" for master's
\documenttitle{Dissertation}

\degreename{Doctor of Philosophy}

% Use the wording given in the official list of degrees awarded by UCI:
% http://www.rgs.uci.edu/grad/academic/degrees_offered.htm
\degreefield{Physics}

% Your name as it appears on official UCI records.
\authorname{Chengcen Sha}

% Use the full name of each committee member.
\committeechair{Professor Ilya Krivorotov}
\othercommitteemembers
{
  Professor Wilson Ho\\
  Professor Jing Xia
}

\degreeyear{2018}

\copyrightdeclaration
{
  {\copyright} {\Degreeyear} \Authorname
}

% If you have previously published parts of your manuscript, you must list the
% copyright holders; see Section 3.2 of the UCI Thesis and Dissertation Manual.
% Otherwise, this section may be omitted.
% \prepublishedcopyrightdeclaration
% {
% 	Chapter 4 {\copyright} 2003 Springer-Verlag \\
% 	Portion of Chapter 5 {\copyright} 1999 John Wiley \& Sons, Inc. \\
% 	All other materials {\copyright} {\Degreeyear} \Authorname
% }

% The dedication page is optional
% (comment out to exclude).
\dedications
{
  To My parents, Zhenglian and Wenyu.
}

\acknowledgments
{
  I would like to express my deepest appreciation to Professor Ilya Krivorotov. It is quite a privilege to work with him. I really learned a lot from his great knowledge of physics and sharp insights into problem solving. I still remember almost six years ago I had my first Skype interview with Ilya and he asked me about exchange bias effect. Without his guidance and help this dissertation would not be possible.
  
  I would also like to thank all my lab mates : Zheng Duan, Igor Barsukov, Eric Montoya, Brian Youngblood, Liu Yang, Yu-Jin Chen, Andrew  Smith,  Han Kyu Lee, Jenru Chen, Jieyi Zhang,  Alejandro Jara, Chris Safranski and Josh Dill. I have been received a great amount of help since the first day I joined the lab. When I first joined the lab, I did not know even most simple thing, such as using a torque wrench and basis soldering. These guys helped me overcome a lot of difficulties and provided help whenever I asked. I never wished I could be surrounded by such a amazing group of people.
    
  During the last five years in the United States, I am happy to keep in touch with my good friends. I would like to thank Jiao Li, Rui Da, Ruohui Yang, Xun Liu, Hongyu Zhu and Qingyu Zhu. Those people are all pursuing or have already obtained their Ph.D. degree and we have had lots of useful/useless discussions about our PhD lifes. We shared numerous joy and distress over the past five years. I might not be able to see these friends quite often and in fact, I have not met some of them for almost five years. However, their support is greatly valuable to me and I really hope nothing but the best for all of them from my heart.
  
  Of course I am extremely luck to have a loving family. My parents have been the greatest support to me without any conditions. They have provided all they can give to me and I cannot pay back them enough. My wife has been the angle of my life and I love her very much. Of course I cannot forget my cat Luca, who gives me lots of happiness when I was lonely.
  
}


% Some custom commands for your list of publications and software.
\newcommand{\mypubentry}[3]{
  \begin{tabular*}{1\textwidth}{@{\extracolsep{\fill}}p{4.5in}r}
    \textbf{#1} & \textbf{#2} \\ 
    \multicolumn{2}{@{\extracolsep{\fill}}p{.95\textwidth}}{#3}\vspace{6pt} \\
  \end{tabular*}
}
\newcommand{\mysoftentry}[3]{
  \begin{tabular*}{1\textwidth}{@{\extracolsep{\fill}}lr}
    \textbf{#1} & \url{#2} \\
    \multicolumn{2}{@{\extracolsep{\fill}}p{.95\textwidth}}
    {\emph{#3}}\vspace{-6pt} \\
  \end{tabular*}
}

% Include, at minimum, a listing of your degrees and educational
% achievements with dates and the school where the degrees were
% earned. This should include the degree currently being
% attained. Other than that it's mostly up to you what to include here
% and how to format it, below is just an example.
%
% CV is required for PhD theses, but not Master's
% comment out to exclude

\curriculumvitae
{

\textbf{EDUCATION}
  
  \begin{tabular*}{1\textwidth}{@{\extracolsep{\fill}}lr}
    \textbf{Doctor of Philosophy in Physics} & \textbf{2018} \\
    \vspace{6pt}
    University of California Irvine & \emph{Irvine, CA, United States} \\
    \textbf{Bachelor of Science} & \textbf{2013} \\
    \vspace{6pt}
    Nanjing University & \emph{Nanjing, China} \\
  \end{tabular*}

\vspace{12pt}
\textbf{RESEARCH EXPERIENCE}

  \begin{tabular*}{1\textwidth}{@{\extracolsep{\fill}}lr}
    \textbf{Graduate Research Assistant} & \textbf{2013--2018} \\
    \vspace{6pt}
    University of California, Irvine & \emph{Irvine, California} \\
  \end{tabular*}

\vspace{12pt}
\textbf{TEACHING EXPERIENCE}

  \begin{tabular*}{1\textwidth}{@{\extracolsep{\fill}}lr}
    \textbf{Teaching Assistant} & \textbf{2013--2014} \\
    \vspace{6pt}
    University of California, Irvine & \emph{Irvine, California} \\
  \end{tabular*}



\vspace{12pt}
\textbf{SELECTED PRESENTATIONS AND POSTERS }

	\mypubentry{Characterization of perpendicular STT-MRAM by ST-FMR}{March 2016}{APS March Meeting}
  \mypubentry{Ferromagnetic Resonance Linewidth in Nanoscale Magnetic Tunnel Junctions }{November 2017}{MMM}
  \mypubentry{Ferromagnetic resonance linewidth in nanoscale magnetic tunnel junctions }{August 2018}{International Conference on Magnetism}
  
}

% The abstract should not be over 350 words, although that's
% supposedly somewhat of a soft constraint.
\thesisabstract
{
 	Spin transfer torque is generated by the transfer of angular momentum from spin polarized electrons to a ferromagnet. This spin transfer torque provides an efficient way to manipulate the dynamic motion of the magnetization of a nanomagnet, and can be strong enough to induce magnetization switching and steady-state precession. This field of study draws enormous attention not only because spin transfer torque is essential in understanding fundamental physical phenomena, but also it makes the building block for future applications such as spin torque oscillators, magnetic random access memory.We have developed several new techniques to characterize such dynamics in nanoscale magnetic tunnel junctions. In this thesis we will first introduce a effect methods to characterize important material parameters in nano-scale Magnetic Tunnel Junctions: Spin-torque ferromagnetic resonance. This methods combing with micromagnetic modeling allows us to determine the magnetic anisotropy, Gilbert damping, exchange stiffness and shape distortion and damages. We will also demonstrate a single-shot electrical technique to capture the magnetic dynamics during the spin torque switching of a magnetic tunnel junction in real time. We also discuss measurement of switching probability of magnetic tunnel junctions by applying electric pulses.
}


%%% Local Variables: ***
%%% mode: latex ***
%%% TeX-master: "thesis.tex" ***
%%% End: ***
