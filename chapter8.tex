\chapter{Determination of Exchange Stiffness of STT-MRAM devices with broken symmetry}

We have developed one of the world most sensitive spin-torque ferromagnetic resonance(ST-FMR)\cite{FieldMod} and we would like to accurately determine the exchange stiffness of the Magnetic Tunnel Junctions. The exchange interaction is essential since it determines the energy scale of two adjacent spins in the magnetic materials. Its value also affect the formation of magnetic structures such as domain walls and vortices. Therefore it is important for both fundamental scientific interest and technology development to measure the exchange stiffness from simple structures such as monolayer superlattices and thin films to complex systems such as the Magnetic Tunnel Junctions(MTJ). It has been demonstrated that the exchange stiffness in in-plane magnetized MTJs can be estimated by measuring the thermal stability factor and fit of model based on nucleation-type magnetization reversal\cite{thermalEx} and by modelling from microwave noise spectroscopy\cite{noiseEx}. It has also been showed that MTJs with perpendicular magnetic anisotropy can be utilized by characterizing the spin wave dispersion to determine the exchange stiffness\cite{chris}\cite{PMAEX}. However previous studies involving MTJs are focused on nominal circular devices and only rely on the mode spacings between first higher order modes and quasi-uniform modes. However the symmetry breaking in the nominal circular devices\cite{excitation2},which is often inevitable during the fabrication of the STT-MRAM devices, has altered the spin wave modes. In this chapter we would like to perform a comprehensive review of determining the exchange stiffness on both nominal circular devices and stadium shaped devices with different lateral dimensions.

\section{Measurement of ST-FMR on nominal circular devices}

The experimental set-up is based on Fig.\ref{fig:FMR_set_up} where we employ the field modulation technique to improve the signal-to-noise ratio. The STT-MRAM devices we measured are CoFeB based Magnetic Tunnel Junctions. We are mainly focused on the field-domain ST-FMR which sweeps the magnetic field at fixed constant driven frequency. We find that the field-domain measurement is usually faster and yields better signal compared with frequency-domain. 

\begin{figure}[!ht]
\centering
\subfigure{\label{fig:C07MR}\includegraphics[width=75mm]{fig/2018/C07MR}}
\subfigure{\label{fig:C07FH}\includegraphics[width=75mm]{fig/2018/C07FH}}
\caption{(a) Example magnetoresistance of one 80 nm MTJ device (b) 2D contour plot of the ST-FMR signal of this device with -2 dBm power applied at the AP state  }
\end{figure}

We start with nominal circular devices with diameter ranging from 70 nm to 210 nm. Fig.\ref{fig:C07MR} shows an example magnetoresistance of 80 nm MTJ device. This specific device has resistance of 2511 Ohms at the parallel state and 5631 Ohms at the anti-parallel state. The coercive field is about 2100 Oe. Fig.\ref{fig:C07FH} shows the 2D contour plot of the ST-FMR signal of this device with -2 dBm power applied at the AP state. From the 2D contour plot we can mainly identify three of the spin wave modes, each of them labelled on the plot. The lowest mode(Mode 0) is the quasi-uniform main mode of the free layer in the MTJs. Mode 1 and Mode 2 is the split first higher order mode(we will discuss the mode profile in the next section).

\begin{figure}[!ht]
  \centering
  \includegraphics[width=0.6\textwidth]{fig/2018/C07fit}
   \caption{Top: the frequency versus resonance field for three lowest modes. Bottom: the HWHM linewidth versus resonance field for all the three modes.  }
  \label{fig:07fit}
\end{figure}

Fig.\ref{fig:07fit} summarizes the mode fitting result for all the three modes we excited from this 80 nm device. The top panel shows the frequency versus resonance field for three lowest modes. We can see a linear relation between the driven frequency and the resonance field as predicted by the Kittel equation. The main mode at zero field is 9.83 GHz and the effective anisotropy field Hk around 3.4 kG. The bottom panel shows the HWHM linewidth versus resonance field for all the three modes. Quite surprisingly, linewidth does not have a strong field dependence from -3000 Oe to 500 Oe (and frequency change by nearly a factor of 4). In theory, the linewidth in the field domain should be given by
\begin{equation}
	\Delta H = \alpha  \frac{\omega}{\gamma} + \Delta H_0
\end{equation}
Here, $\Delta H$ is the field-domain HWHM linewidth. $\omega = 2 \pi f$ is the angular frequency. $\gamma / 2 \pi$ is the gyromagnetic ratio. Firstly, this non-linear relation reveals that the fundamental understanding of the large non-Gilbert contribution to the damping is lacking. Secondly, we find that the linewidth for this device is relatively small (around 35 Oe). If we use the zero-field linewidth as upper bound, we have an estimation of Gilbert damping around 0.01. 


So far we have demonstrated that by performing the ST-FMR measurements, we can determine the "resonance frequency" at zero magnetic field for all the modes excited in the experiment. The frequency of the main mode can be used to determine the effective anisotropy field and the mode spacings between the main mode and the higher order mode is related with the exchange stiffness of the free layer. Before we move to the exchange stiffness, let us first discuss the original of the ST-FMR signal in this perpendicular magnetized MTJs.




\clearpage


\section{Summary of Circular Devices: Experimental Data}

Now we would like to summarize the experimental data of circular devices with different diameters. As we have demonstrated in Fig.\ref{fig:C07FH}, the signature of the spin wave modes excited in these devices are one lowest quasi-uniform main mode(Mode 0) with two split higher-order modes(Mode 1 Mode 2). For each dimension, we can measure over ten devices to obtain mode statistics. The idea of measuring nominally identical devices is to reduce random sample-to-sample variations. For each measured device, we can list the frequencies of three spin-wave modes at zero magnetic field. These raw data can be found from the appendix. From the mode statistics we can extract the average mode frequency and standard deviations.

\begin{figure}[!ht]
  \centering
  \includegraphics[width=0.6\textwidth]{fig/2018/ModevsSize}
   \caption{Summary of main mode of all the three modes. The error bar indicates the standard deviations obtained from sample statistics.}
  \label{fig:ModeVsSize}
\end{figure}

The summary of main mode and standard deviations of all the three modes are plotted in Fig.\ref{fig:ModeVsSize}. As the device diameter goes up, the mode frequencies of three modes reduces due to shape anisotropy reduction.

Now let us first focus on the main mode and plot it as a function of device diameter as shown in Fig.\ref{fig:GapvsSize}. The effective demagnetization factor is given by\cite{demagfactor}.

\begin{equation}\label{eq:Nz}
	\centering
	N_z \approx 1- \frac{1}{\pi d} [2 \ln(4 \frac{d}{t} ) -1 ]
\end{equation}

where the d is the device diameter and t is the free layer thickness. From Eq.\ref{eq:Nz} and the fact that $N_x + N_y + N_z = 1$, the total perpendicular anisotropy $H_{ku}$ can be written as\cite{Kittel}

\begin{equation}\label{eq:Hk}
	\centering
	H_k = H_{ku} + 2 \pi (1 - 3 N_z) M_s
\end{equation}

Assuming the free layer thickness 1.6 nm, the fitted result  $M_s \ 1820 \ \text{emu}/cm^3$ and  $H_{ku} \ 24 \ \text{KOe}$, which is comparably with other independent measured values.

\begin{figure}[!ht]
\centering
\subfigure{\label{fig:GapvsSize}\includegraphics[width=75mm]{fig/2018/GapvsSize}}
\subfigure{\label{fig:LinearFit}\includegraphics[width=75mm]{fig/2018/LinearFit}}
\caption{(a) Main mode frequency is plotted as a function of 
device diameter. (b) Mode Gap plotted as a function $1/d^2$ with d represents the diameter of the device.}
\end{figure}



Fig.\ref{fig:LinearFit} shows mode gap plotted as a function $1/d^2$ with d represents the diameter of the device. The mode gap between first two modes in a circular device can be modeled as
\begin{equation}
\label{eq:wave}
	\hbar (\omega_1 - \omega_0 ) = D (s/d)^2
\end{equation}

Here $\omega = 2\pi  f$ represents the frequency of the mode. D is the exchange stiffness which is related with exchange constant $Aex$ by $A_{ex} = \frac{D M_s}{2 g \mu_B} $. (g: g-factor. $M_s$ saturation magnetization. $\mu_B$ Bohr magneton). s is a numerical factor which is close to 3.68. By perform a linear fit of Fig.\ref{fig:LinearFit} we can obtain the $A_{ex}$ value of 8.9 pJ/m, which is reduced from the bulk value around 15 pJ/m.

\clearpage


\section{Micromagnetic Simulations of the Mode Spacing}

After we obtained reliable experimental data, we can use Micromagnetic simulation to determine the exchange stiffness based on the mode spacings. Fig.\ref{fig:simulated} shows the simulated Magnetic Tunnel Junctions(MTJs) with magnetization and thickness for each active layer. The free layer and SAF top layer magnetization are obtained from independent measurement results and the SAF Bottom layer magnetization is determined by simulating the magnetization versus external magnetic field curve so that the center of the magnetization curve is close to zero. This is because of the balanced SAF layer dipolar field confirmed by the experiment.

\begin{figure}[!ht]
  \centering
  \includegraphics[width=0.8\textwidth]{fig/2018/simulated}
   \caption{Simulated MTJs structures. The magnetization and thickness parameters are listed for each magnetic layer.}
  \label{fig:simulated}
\end{figure}

In the micromagnetic simulations, we use the perpendicular magnetic anisotropy to fit for the frequencies of the main mode. The mode gap is ideally determined only by the exchange stiffness. Fig.\ref{fig:7070} shows a typical simulated spectrum for 70 nm diameter circular devices with the magnet anisotropy $10.5*10^5 J/m^3$ and the exchange stiffness $ A_{ex} \; 12 \; pJ/m $. The mode profiles of these two modes are listed around the spectrum peak. The main mode has uniform amplitude with small variations around the edges. The first higher-order mode has a node at the center of the circle. As we identified previously, the nominal circular devices have common shape distortions so that the actual size is not perfectly circular. In the next step of the simulations, we introduce such shape distortions to make a ellipse MTJ device. Fig.\ref{fig:ellipse} shows the simulated spectrum of ellipse devices with different degree of symmetry breaking. As we vary the shape distortion, the frequency of the main mode does not change too much, and the first order higher mode splits into two modes(mode 1 and 2). As the ellipticity increases, the frequency of the lower mode 1 decreases while the frequency of the higher mode 2 increases. Fig.\ref{fig:shape} shows the splitting of the first order mode(with node in the center) into two separate modes with nodes along the long and short axis of the ellipse.

\begin{figure}[!ht]
\centering
\subfigure{\label{fig:7070}\includegraphics[width=50mm]{fig/2018/sim/70_70}}
\subfigure{\label{fig:ellipse}\includegraphics[width=50mm]{fig/2018/sim/ellipse}}
\subfigure{\label{fig:shape}\includegraphics[width=50mm]{fig/2018/sim/split}}
\caption{(a) Simulated 70 nm diameter spectrum with mode profiles showing around the peak. (b) Simulated spectrum with different elliptical shapes. (c) The mode profile splitting of the first higher order mode into two modes in the ellipse with nodes along short and long axis.}
\end{figure}

As we mentioned previously, the mode spacings between the main mode and the first higher order mode are ideally determined by the exchange stiffness. The question we need to answer now is how we define the mode spacing with shape distortions. Table.\ref{shapedist} summarizes the simulated mode frequencies with different geometries and comparisons with the experimental data. As we can see from the table, as we increase the ellipticity, the mode gap (1+2)-0, which is defined as the average of mode 1 and mode 2 minus the mode 0, is nearly constant.


\begin{table}[ht!]
\centering
\begin{tabular}{|l|l|l|l|l|l|}
\hline
\textbf{} & \textbf{70*70} & \textbf{75*65} & \textbf{85*60} & \textbf{90*55} & \textbf{Experimental} \\ \hline
\textbf{Mode 0} & 11.74 & 11.82 & 11.78 & 11.9 & 11.55 \\ \hline
\textbf{Mode 1} & 14.5 & 14.46 & 14.06 & 14.04 & 13.94 \\ \hline
\textbf{Mode 2} &  & 14.9 & 15.2 & 15.72 & 14.6 \\ \hline
\textbf{Gap (1+2)/2-0} & 2.76 & 2.86 & 2.85 & 2.9 & 2.72 \\ \hline
\textbf{Gap 2-1} & 0 & 0.44 & 1.14 & 1.68 & 0.66 \\ \hline
\textbf{Aspect ratio} & 1 & 1.15 & 1.41 & 1.63 &  \\ \hline
\end{tabular}
\caption{The simulated mode frequencies with different geometries and comparisons with the experimental data. All the numbers are in GHz unit.}
\label{shapedist}
\end{table}

Now we can summarize the procedure of the mode spacing fitting for these nominal circular devices. We will use the average of mode 1 and 2 from the experiment to fit for the mode spacings between the simulated mode 1 and 0 in circular device spectrum. It should be noted that the shape distortions can be random across the devices and the actual shape of each device can be different. This will create unavoidable uncertainty in the exchange stiffness fitting.


\begin{figure}[!ht]
  \centering
  \includegraphics[width=0.8\textwidth]{fig/2018/sim/55sim}
   \caption{Top: Simulated mode frequencies for the first two modes excited in the 55 nm diameter circular devices as a function of the exchange constant. Bottom: The mode spacings as a function the exchange constant with linear fitting.}
  \label{fig:55sim}
\end{figure}

Before we move to detailed micromagnetic simulations, we need to consider the actual size of the MTJs. In the MTJ fabrication process, we find that the diameter of the MTJs in this batch is typically 14.4 nm smaller than the listed diameter. For nominal 70 nm devices, the input diameter is 55 nm(an integer number of micromagnetic cell size). The top of Fig.\ref{fig:55sim} shows the simulated 55 nm frequencies of the main mode and first higher order mode as a function of input exchange stiffness. The frequencies of the two modes are increasing linearly with the exchange stiffness. The bottom of Fig.\ref{fig:55sim} shows the mode spacings between the first two modes as a function of $A_{ex}$, which can be then fit by a line. The linear dependence of the mode spacings on the exchange stiffness is expected as we discussed in the Eq.\ref{eq:wave}. 

\begin{table}[ht!]
\centering
\begin{tabular}{ c c }
      \addheight{\includegraphics[width=60mm]{fig/2018/sim/65sim}} &
      \addheight{\includegraphics[width=60mm]{fig/2018/sim/75sim}} \\
      \addheight{\includegraphics[width=60mm]{fig/2018/sim/105sim}} &
      \addheight{\includegraphics[width=60mm]{fig/2018/sim/135sim}} \\
\end{tabular}
\caption{Circular Device Simulation Summary. For different sizes, Both the frequencies of the two modes and the mode spacing are plotted against the exchange constant}
\label{circularsummary}
\end{table}

Based on this linear relation, we can determine the best-fit exchange stiffness from the experimental value of mode spacings. Moreover, we can also obtain the standard deviations of the fitted exchange value based on the standard deviations of the mode spacings from the experiment. By varying the diameters of the devices, we can repeat this fitting procedure for different sizes as listed in the Table.\ref{circularsummary}. Typically we find that by varying the device diameter, the mode spacing has similar linear trend versus the exchange stiffness. The linear slope, however, is decreasing as the device diameter increases. Since the linear slope gives the sensitivity of the exchange stiffness fitting, the relatively larger devices mode spacings are less sensitive to the exchange, which introduces larger fitting variations as we will see.

\begin{figure}[!ht]
  \centering
  \includegraphics[width=0.6\textwidth]{fig/2018/sim/circular_summary}
   \caption{Circular Device Simulation Summary. For each nominal device, we list the simulated diameter(15 nm smaller). The exchange stiffness and the standard variations are listed.}
  \label{fig:circularsummary}
\end{figure}

Fig.\ref{fig:circularsummary} summarizes the exchange stiffness fitting for different geometries. The values are listed in the Table.\ref{cirexsummary}. The actual simulated diameters are 15 nm smaller than the nominal size. The exchange stiffness is determined by the experimental mode spacings and simulated linear relations between mode spacings and simulation exchange stiffness. The standard deviations are obtained from the standard deviations of the mode spacing. 

\begin{table}[]
\centering
\begin{tabular}{|l|l|l|l|}
\hline
\textbf{Nominal Device length (nm)} & \textbf{Device Length* (nm)} & \textbf{Aex (pJ/m)} & \textbf{Error (pJ/m)} \\ \hline
\textbf{70} & 55 & 6.4 & 2 \\ \hline
\textbf{80} & 65 & 7.6 & 2 \\ \hline
\textbf{90} & 75 & 8.8 & 3 \\ \hline
\textbf{120} & 105 & 13.8 & 5 \\ \hline
\textbf{150} & 135 & 14.1 & 3 \\ \hline
\end{tabular}
\caption{Summary of fitted exchange stiffness for different diameters.}
\label{cirexsummary}
\end{table}

There are two major conclusions here. First, we find that for all the five diameters there are reductions of the exchange stiffness compared with the bulk value of the Co and Fe thin films, which is in the range of 15 pJ/m and 25 pJ/m. The reductions of the exchange stiffness from the thin films to the MTJs devices have been observed from our collaborators and from our previous continuous FMR simulations as well. One possible origin is the formation of different grains within the MTJs free layer and the reduction of inter-grain exchange couplings.


\begin{figure}[!ht]
  \centering
  \includegraphics[width=0.6\textwidth]{fig/2018/sim/free_only_circular}
   \caption{Simulations of the mode spacings as a function of the exchange stiffness for different diameters. The MTJs only has a magnetic free layer.}
  \label{fig:freecircularsummary}
\end{figure}

Second, we find that the exchange stiffness is increasing as the device diameter increases. We have not fully understood this behavior at this time. However, as we mentioned earlier, the mode spacings of the larger devices are less sensitive to the exchange stiffness. Moreover, since the determination of the exchange stiffness depends on the actual geometries, if the larger devices deviate more from the nominal size, it would create more error in the exchange stiffness fitting.

Before we move to other geometries, we find that the mode spacings do not go to zero as the exchange stiffness shown in the Fig.\ref{fig:circularsummary}. If we would only consider the exchange couplings between the free layers, the mode spacings should go to zero at zero $A_{ex}$. To understand this behavior, we perform the same simulation without the SAF layers in the MTJs as shown in the Fig.\ref{fig:freecircularsummary}. Only the dynamics of the free layer is considered in this simulation. We also plot the mode spacings as a function of the $A_{ex}$ for different diameters. We find that the mode spacings go to zero as the $A_{ex}$ goes to zero in this free-layer-only simulation.  So we can conclude that the non-zero intercept of the mode spacings at zero $A_{ex}$ is likely due to the dipolar field from the SAF layers on the free layer.

\section{Study of signal amplitude of ST-FMR signal}

While we have not discussed the origin of the ST-FMR signal in the perpendicular magnetized MTJs with magnetic field applied in the easy axis, the exact source of the signal is not very clear. In fact, in a ideal circular device with rotation symmetry, such arrangement should yield no DC self rectification since the spin transfer torque is zero with free layer pinned in the perpendicular direction. However, we can still detect a measurable signal out of this set-up. It is argued that there are local misalignments between the uniaxial anisotropy and applied magnetic field due to shape distortion\cite{chris}. It is also possible that the non-uniformity of spin-transfer torque from the tunnel current across the free layer and non-uniform  tunnel magnetoresistance (TMR) should contribute to the ST-FMR signal. Previous work has been done to quantitatively measure the spin-transfer-torque in the ST-FMR signal\cite{Sankey2008}\cite{Kubota2007}\cite{NetSTT} for the MTJs with in-plane easy axis. At the same time, Ab initio studies has been done to study the spin-transfer torque in MTJs\cite{AbInitio}. However, in our field-modulated ST-FMR set-up, such experiments are not accessible since we are actually measuring the field derivatives of the real signal. Nevertheless, we still would like to qualitatively measure the signal amplitude of our ST-FMR data and try to gain some knowledge out of it.


\begin{figure}[!ht]
\centering
\subfigure{\label{fig:C10MR}\includegraphics[width=50mm]{fig/2018/C10MR}}
\subfigure{\label{fig:C10RDC}\includegraphics[width=50mm]{fig/2018/C10RDC}}
\subfigure{\label{fig:C102D}\includegraphics[width=50mm]{fig/2018/C102D}}
\caption{(a) Example magnetoresistance of one 60 nm * 80 nm stadium-shaped MTJ device. (b) The resistance versus current loop, with AP state switching to P state at negative dc current and vice versa. (c) 2D contour plot of the ST-FMR signal of this device with -2 dBm power applied at the AP state }
\end{figure}

The device employed in this study has a stadium shape with lateral dimensions 60 nm * 80 nm. The introduced broken symmetry away from perfect circular is designed to promote better ST-FMR signal. Fig.\ref{fig:C10MR} shows the magnetoresistance of this device, which has resistance of 3445 Ohms at the parallel state and 7800 Ohms at the anti-parallel state. The coercive field is about 1850 Oe. Fig.\ref{fig:C10DC} shows the resistance versus current loop. The negative(positive) current is the anti-damping polarity for AP to P(P to AP)state. This demonstrate the effect of spin-transfer-torque switching. Fig.\ref{fig:C102D} shows the 2D contour plot of the ST-FMR signal of this device with -1 dBm power applied at the AP state. Again we can mainly identify three of the spin wave modes. 

\begin{figure}[!ht]
\centering
\subfigure{\label{fig:C10FH}\includegraphics[width=50mm]{fig/2018/C10fit}}
\subfigure{\label{fig:C10signal}\includegraphics[width=50mm]{fig/2018/C10signal}}
\subfigure{\label{fig:C10ratio}\includegraphics[width=50mm]{fig/2018/C10ratio}}
\caption{(a) The resonance field fitting result for all the three modes. (b) The ST-FMR signal (symmetric and anti-symmetric component) plot versus resonance field. (c) The ratio of two components versus resonance magnetic field. }
\end{figure}

Fig.\ref{fig:C10FH} summarizes the mode fitting result for all the three modes we excited from this stadium shape device. The linear relation is reproduced as expected. The main mode at zero field is 10.81 GHz and the effective anisotropy field Hk around 3.7 kG. Fig.\ref{fig:C10signal} shows the ST-FMR signal (symmetric and anti-symmetric component) plot versus resonance field. The amplitude is larger at small field and smaller at negative large magnetic field. This is due to magnetic susceptibility as expected. As we can see from Fig.\ref{fig:C10MR} that the MTJ at the AP state aligns better with the external negative magnetic field, which gives smaller ST-FMR amplitude. Fig.\ref{fig:C10ratio} shows the ratio of two components versus resonance magnetic field. And we see that the ratio remain relatively unchanged over the resonance field. The ratio of this two components is related with different signal mechanism which will be discussed later.

\begin{figure}[!ht]
\centering
\subfigure{\label{fig:C10power}\includegraphics[width=75mm]{fig/2018/C10power}}
\subfigure{\label{fig:C10poweramp}\includegraphics[width=75mm]{fig/2018/C10powersignal}}
\caption{(a) The power-dependent ST-FMR field sweep spectrum at AP state with 12 GHz. (b) The amplitude versus the square of rf voltage.}
\end{figure}

The applied power determines the rf voltage across the MTJs. Fig.\ref{fig:C10power} shows the power-dependent ST-FMR field sweep spectrum at AP state with 12 GHz. We can then plot the amplitude versus the square of rf voltage as shown in Fig.\ref{fig:C10poweramp}. The good linear fit indicating the ST-FMR signal mainly arises from rectification \cite{photovoltage1}. As the rf voltage approaches zero, both symmetric and anti-symmetric components goes to zero at a similar value. In fact, if we apply zero dc bias into the MTJ, this limit will go to zero at zero rf voltage. Thus we can confirm there is also a non-zero contribution from the photo-resistance effect\cite{photovoltage2}.


\begin{figure}[!ht]
\centering
\subfigure{\label{fig:C10powerfit}\includegraphics[width=75mm]{fig/2018/C10powerFH}}
\subfigure{\label{fig:C10powerLW}\includegraphics[width=75mm]{fig/2018/C10powerLW}}
\caption{(a) The resonance Field versus rf voltage squared  for all the modes at 12 GHz. (b) The Mode 0 HWHM linewidth versus rf voltage squared.}
\end{figure}

Fig.\ref{fig:C10powerfit} shows the resonance Field versus rf voltage squared  for all the modes at 12 GHz and Fig.\ref{fig:C10powerLW} shows the Mode 0 HWHM linewidth versus rf voltage squared. The linewidth shows a linear dependence with zero rf voltage linewidth around 23 Oe. The linear relation of both resonance field and linewidth ensures that we keep the ST-FMR measurement at the linear region where there is no non-linear broadening of the signal.
 
\begin{figure}[!ht]
\centering
\subfigure{\label{fig:C10DC}\includegraphics[width=75mm]{fig/2018/C10DC}}
\subfigure{\label{fig:C10DCamp}\includegraphics[width=75mm]{fig/2018/C10DCsignal}}
\caption{(a) The bias-dependent ST-FMR field sweep spectrum at AP state with 12 GHz. (b) FMR signal amplitude versus applied dc voltage. }
\end{figure}

After we vary the applied the power in the ST-FMR measurement, it is also interesting to change the dc bias. Fig.\ref{fig:C10DC} shows the bias-dependent ST-FMR field sweep spectrum at AP state with 12 GHz. All the three modes have the same curvature under external bias, which proves that these are all the spin-wave modes from the free layer. Fig.\ref{fig:C10DCamp} shows the FMR signal amplitude versus applied dc voltage. We find that the symmetric component(relating to spin transfer torque) shows a quadric+linear dependent and the anti-symmetric component was nearly invariant versus bias.

\begin{figure}[!ht]
  \centering
  \includegraphics[width=0.6\textwidth]{fig/2018/C10DCFIT}
   \caption{Fitting of the FMR symmetric amplitude versus DC voltage as a sum of linear term and quadric term.}
  \label{fig:C10DCfit}
\end{figure}

We can further analyze the FMR symmetric amplitude versus DC voltage as shown in Fig.\ref{fig:C10DCfit}. The Symmetric amplitude shows as sum of linear term and quadric term. The linear term, related to the photo-resistance contribution, changes the sign with different polarity of dc voltage. Moreover, the linear term is nearly dominated except at large dc voltage, which indicating that the photo-resistance effect is not negligible in the system!

\begin{figure}[!ht]
  \centering
  \includegraphics[width=0.4\textwidth]{fig/2018/demo}
   \caption{Demo of Perpendicular magnetized MTJs angle oscillation with out-of-plane magnetic field.}
  \label{fig:demo}
\end{figure}

The ST-FMR signal of MTJs with PMA can be modeled as shown in the Fig.\ref{fig:demo}. In the MTJs, there is a small angle $\theta_{mis}$ between the free layer and the fixed layer. The magnetic field H is applied in the perpendicular direction. After apply dc bias and microwave power, the resistance of the MTJ can be written as 
	\begin{equation}
		R = R_{dc} + R_{ac} \sin{\omega t} + R_{20} \sin{2 \omega t} 
	\end{equation}
	
Here the $R_dc$ is the time-averaged MTJ resistance and $R_ac$ is the oscillating MTJ resistance. $R_{20}$ is the higher harmonic component of the resistance. The amplitude the angular oscillation is $\theta_A$. After we define those variables, the ST-FMR signal can be expressed as 
\begin{equation}
	V_{ST-FMR} \propto I_{dc}R_{dc} + \frac{1}{2} <I_{ac}R_{ac}>
\end{equation}

The first term is related with contributions from photo-voltage(related to rectification) and the second term is related with photo-resistance(related to time-averaged resistance).


\begin{figure}[!ht]
\centering
\subfigure{\label{fig:C10DCFH}\includegraphics[width=75mm]{fig/2018/C10DC-FH}}
\subfigure{\label{fig:C10DCLW}\includegraphics[width=75mm]{fig/2018/C10DC-LW}}
\caption{(a) The resonance field versus bias for three modes. (b) Mode 0 HWHM linewidth versus applied dc bias.}
\end{figure}

Before we end this section, we can also fit the spin-wave modes under finite dc bias. Fig.\ref{fig:C10DCFH} shows the resonance field versus bias for three modes. The main mode has both linear and quadratic dependence. The quadratic dependence is harder to analyze since it is a mixed contribution from field-like torque and ohmic heating. The linear term is believed to relate with voltage-controlled magnetic anisotropy(VCMA). The linear slope gives VCMA 244 Oe/V, close to previously measured circular devices. Fig.\ref{fig:C10DCLW} shows Mode 0 HWHM linewidth versus applied bias. At positive voltage(damping), we find the linewidth increase as increasing voltage(as expected). At negative voltage(anti-damping),however, the linewidth unexpectedly increase with larger negative voltage. This is also a strong evidence of non-linear damping in this type of devices. One possible explanation is that, when approaching the switching region, the free layer has more fluctuations which contributes to the linewidth broadening.



\clearpage

\section{Summary of Stadium shapes MTJs devices with 60 nm width}

As we have measured circular devices with different diameters, for stadium devices, we also have measured devices with nominal 60 nm width and length from 70 nm to 210 nm. Before we move to the mode statistics, we first check the mode structures for these 60 nm devices. Fig.\ref{fig:210nmdata}(a) shows the Magneto-resistance curve for one typical 210 nm * 60 nm device. Fig.\ref{fig:210nmdata}(b) shows the ST-FMR field sweep 2D spectrum at AP state with example scan at 12 GHz. From the sample trace at 12 GHz, we find that for these larger devices, although the mode density becomes large and there are more higher order modes with smaller mode spacings, we still reproduce the general mode structures as the circular devices: one main mode at the lowest frequency and two first higher-order modes.  


\begin{figure}[!ht]
  \centering
  \includegraphics[width=1.0\textwidth]{fig/stadium/210nmdata}
   \caption{(a) Magneto-resistance curve for 210 nm * 60 nm devices (b) ST-FMR field sweep 2D spectrum at AP state with example scan at 12 GHz.}
  \label{fig:210nmdata}
\end{figure}

Since we have reproduced the similar mode profiles as circular devices, we will use the same mode spacing as the circular devices to compare with the micromagnetic simulations: the average of the first two higher order modes minus the lowest main mode. To obtain the experimental spacings for stadium-shape devices with different length, we first performed detailed ST-FMR measurements and obtained good mode frequencies statistics. The summary of the stadium devices with 60 nm nominal width and varying length from 70 nm to 210 nm is listed in Fig.\ref{fig:60modesummary}.


\begin{figure}[!ht]
  \centering
  \includegraphics[width=1.0\textwidth]{fig/stadium/60modesummary}
   \caption{Main mode frequency with standard deviations versus device length for “60” width devices. The plotted device length is 15 nm smaller than nominal size}
  \label{fig:60modesummary}
\end{figure}
 
We plot the main mode frequency with standard deviations versus device length for nominal 60 nm width devices. The plotted device length is 15 nm smaller than nominal size. The main mode frequency drops over 15 per cent, which is due to shape anisotropy reductions for larger devices.


\begin{figure}[!ht]
  \centering
  \includegraphics[width=1.0\textwidth]{fig/stadium/60nmgapsummary}
   \caption{Mode gap with standard deviations versus device length for nominal 60 nm width devices. The mode gap is decreasing as increasing length.}
  \label{fig:60nmgapsummary}
\end{figure}

In the next step, we plot the mode gap with standard deviations versus device length for nominal 60 nm width devices. The mode gap is decreasing as increasing length. The mode gap between mode 0 and 1 gives a nice 1/length fit in the limit of large aspect ratio as it plotted in the dash line. We can conclude that the 1/length dependence points to the importance of dipole-dipole contributions to the gap values and tt should be possible to extract the exchange parameter from this 1/length dependence. We think that a more advanced fitting function is needed to cover the smaller aspect ratios.

\subsection{Micromagnetic simulations of stadium devices}

We use similar full-stack MTJ structures as shown in Fig.\ref{fig:simulated} to perform micromagnetic simulations to determine the exchange stiffness. To account for the real shape, we also reduce 15 nm from the nominal value in the simulations. Fig.\ref{fig:4565sim} shows the Example of simulation spectrum for 45*65 $nm^2$ device (nominal dimension: 60*80 $nm^2$) with mode profile showing around the peak. For this stadium shape device, there are clear two higher order modes with node along short and long axis. Two key parameters used in Fig.\ref{fig:4565sim} are magnet anisotropy $10.05*10^5 J/m^3$ and the exchange stiffness $ A_{ex} \; 7 \; pJ/m $. The simulated mode frequencies and mode spacings are listed to be compared with the experimental data. We can see that we already have a relatively good fitting result(keep in mind that $ A_{ex} \; 7 \; pJ/m $ is the average value for circular devices).

\begin{figure}[!ht]
  \centering
  \includegraphics[width=1.0\textwidth]{fig/stadium/4565sim}
   \caption{Example of simulation spectrum for 45*65 $nm^2$ device.}
  \label{fig:4565sim}
\end{figure}

One advantage of these stadium shape devices is that the mode structures is quite straightforward and we can reproduce the experimentally measured lowest three modes without worrying about the shapes distortions in the circular devices. Fig.\ref{fig:4565simsum} shows the experimental mode frequencies compared with simulation values in different exchange stiffness (unit in pJ/m) and Fig.\ref{fig:4565gapsim} shows the experimental mode frequencies compared with simulation values in different exchange stiffness (unit in pJ/m). As we know, the separation between mode 1-2 is susceptible to device dimension variations, so we still used the mode spacings between the average of mode 1 and mode 2 and mode 0 to determine the exchange stiffness.

\begin{figure}[!ht]
\centering
\subfigure{\label{fig:4565simsum}\includegraphics[width=75mm]{fig/stadium/4565simsummary}}
\subfigure{\label{fig:4565gapsim}\includegraphics[width=75mm]{fig/stadium/4565simgap}}
\caption{(a) Experimental mode frequencies compared with simulation values in different exchange stiffness (unit in pJ/m).
 (b) Experimental mode spacing compared with simulation values in different exchange stiffness (unit in pJ/m). }
\end{figure}


Fig.\ref{fig:4565ex} shows the mode gaps plotted versus simulated exchange stiffness for 45*65 $nm^2$ devices. Compared with circular devices we talked before, we also identify a clear linear dependence between mode spacings and $A_{ex}$. Similarly, we find that all the mode gaps do not extrapolate to zero in the limit of zero $A_{ex}$. This is because of the dipole-dipole interactions, which is reducing the sensitivity of this methods for finding the exchange stiffness.

\clearpage

\begin{figure}[!ht]
  \centering
  \includegraphics[width=0.6\textwidth]{fig/stadium/4565ex}
   \caption{Mode Gap plotted versus simulated exchange stiffness for 45*65 $nm^2$ device.}
  \label{fig:4565ex}
\end{figure}

We can then vary the length of the devices at fixed width 45 nm and obtain the $A_{ex}$ as a function of device length as showing in Fig.\ref{fig:4565ex} with numbers listed in the table. We find that for relatively smaller devices, from 70 nm to 120 nm length, the exchange stiffness fitted is quite close: around 7 pJ/m with similar standard error. Keep mind that the standard error is obtained from the variations of the mode spacings.

\begin{figure}[!ht]
  \centering
  \includegraphics[width=1.0\textwidth]{fig/stadium/60nmexsummary}
   \caption{Mode Gap plotted versus simulated exchange stiffness for 45*65 $nm^2$ device.}
  \label{fig:4565ex}
\end{figure}

However, for larger devices, we find that there is a strong increase of exchange stiffness for 150 nm devices. It is not clear the origin of this outlier point but we believe it is an artifact for this particular geometry. First of all by performing in-plane ST-FMR measurements, we make sure that there is no missing modes and the mode identification is correct. Secondly, we also find that resistance-area product for this 45-nm-width group does not any anomaly for larger devices. So the outlier only has two possible origins: one is that the dipole-dipole interaction for larger devices is more pronounced, which reduces the reliability of our methods of exchange stiffness fitting. The second one is that the real size of larger devices is not the value we used in the simulations. If the devices size is not nearly 15 nm smaller than the nominal value, we can not trust the values we obtained from the micromagnetic simulations.

\clearpage


\section{Summary of stadium shapes MTJs devices with 45 nm width}

To complete our studies of stadium shapes, we have measured devices with constant nominal width 45 nm and varying length from 60 nm to 120 nm. As usual the actual size of the devices is roughly 15 nm smaller than the nominal values. So for this 45 nm width group of devices, the aspect ratio is larger than the previous devices we have measured. The aspect ratio is greater than 1.5 and we will the how it can be reflected in the mode structures.

\begin{figure}[!ht]
\centering
\label{fig:452D}
\subfigure{\label{fig:4560FH}\includegraphics[width=80mm]{fig/45nm/4560}}
\subfigure{\label{fig:4570FH}\includegraphics[width=80mm]{fig/45nm/4570}}
\caption{(a) Examples of 2D ST-FMR contour plot for nominal 45*60 $nm^2$ device. (b) Examples of 2D ST-FMR contour plot for nominal 45*70 $nm^2$ device.}
\end{figure}


We show the example 2D ST-FMR contour plots for these 45 nm devices in the above. The sample shown in Fig.\ref{fig:4560FH} is 45*60 $nm^2$ and Fig.\ref{fig:4570FH} shows a 45*70 $nm^2$ device. The major difference for devices with 45 nm width is that the mode spacing between mode 1 and mode 2 is quite large. Remember that for this group of samples, the actual aspect ratio is greater than 1.5 so that the energy barrier between these two higher order modes are larger. In the experiment, we usually find that we can only excite the mode 1 clearly for these higher aspect ratio devices. For example, in Fig.\ref{fig:4560FH} there is only mode 1 visible in the plot and in Fig.\ref{fig:4570FH}, the mode 2 can only be seen faintly and can be not fitted in detail.

\begin{figure}[!ht]
\centering
\subfigure{\label{fig:45main}\includegraphics[width=80mm]{fig/45nm/45nmmain}}
\subfigure{\label{fig:45gap}\includegraphics[width=80mm]{fig/45nm/45nmgap}}
\caption{(a) Average main (0) mode frequency (with standard deviations) plotted versus “actual” device length (b) Mode Gap 1-0 shows a 1/length fit }
\end{figure}

One advantage of these high aspect ratio device is that good separations of between higher-order modes allows us to use the mode gap between main mode and mode 1 to determine the exchange stiffness without worrying about shape distortions. For devices with each length, we measure about ten samples to obtain mode statistics. Fig.\ref{fig:45main} shows the average main (0) mode frequency (with standard deviations) plotted versus “actual” device length. We also observed a clear reduction of main mode as increasing length due to decreasing shape anisotropy. Fig.\ref{fig:45gap} shows the mode spacings between the first two modes plotted versus the device length. The mode spacing also shows a good 1/length fit which indicates the dipolar-dipolar interaction is not negligible in the system.

\begin{figure}[!ht]
\centering
\subfigure{\label{fig:45RA}\includegraphics[width=80mm]{fig/45nm/45RA}}
\subfigure{\label{fig:45FA}\includegraphics[width=80mm]{fig/45nm/45FA}}
\caption{(a) AP state main mode frequency plotted versus AP state Resistance. (b) Mode Gap frequency plotted versus AP state Resistance }
\end{figure}

Previously we mentioned that it is important to know the real dimension of the device in the exchange stiffness fitting. One good indication of the device geometry is the resistance of the device. Fig.\ref{fig:45RA} shows the AP state main mode frequency plotted versus AP state resistance for this group of devices. Clearly it shows correlation between main mode frequency and AP state resistance as we expected. For larger devices, the area is larger and the resistance is smaller, at the same time the main mode frequency is smaller as we discussed. Fig.\ref{fig:45FA} shows mode Gap frequency plotted versus AP state resistance. The frequency gaps also scales approximately linearly with AP state resistance for the stadium shaped devices. We also find that in the limit of zero AP resistance, the mode gap frequency also goes to zero as expected. So we can fit for the mode gap based on the resistance which is much easier to measure compared with ST-FMR measurement.

\begin{figure}[!ht]
  \centering
  \includegraphics[width=0.6\textwidth]{fig/45nm/45nmsim}
   \caption{mode gap plotted versus simulated exchange stiffness for different geometries with same 30 nm width}
  \label{fig:45nmsim}
\end{figure}

Based on the experimental data we obtained from Fig.\ref{fig:45main} and Fig.\ref{fig:45gap}, we can perform similar micromagnetic simulations to determine the exchange stiffness. The only difference compared with previous simulations is that we now use the mode spacing between the main mode and the first higher order mode for comparisons between experimental data and simulation outputs. Fig.\ref{fig:45nmsim} summarizes the mode gap plotted versus simulated exchange stiffness for different geometries with same width. Firstly, we find that the mode gap between mode 1-0 does not go to zero when approaching zero $A_{ex}$. This will agrees with our consumptions of non-negligible dipolar-dipolar interactions. Secondly, the slope of mode spacings versus the input exchange stiffness is steeper for nearly circular devices. This enables us to have more precise determination of the exchange stiffness. 

\begin{figure}[!ht]
  \centering
  \includegraphics[width=1.0\textwidth]{fig/45nm/45nmex}
   \caption{Summary of exchange stiffness fitting: size dependence of width 45 nm}
  \label{fig:45nmex}
\end{figure}

Fig.\ref{fig:45nmex} summarizes the exchange stiffness fitting result for devices with same width(nominal 45 nm) and varying length. Compared with nearly circular devices, we have much less standard deviations thanks to improved sensitivity. For this group of devices, the fitted $A_{ex}$ is near 5  pJ/m. The similar reductions of exchange stiffness from bluk values has also been reported\cite{AexJapan}, which might originates different composition and thickness of CoFeB used in the free layer as well as annealing conditions\cite{DomainCoFeB}. Furthermore, the fact that we have better fitting results for this group of higher aspect ratio devices that the nearly circular devices might not be best for study of $A_{ex}$ due to shape distortions and eigenmodes mixing.
