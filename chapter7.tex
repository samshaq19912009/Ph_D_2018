\chapter{Magnetic field dependence of spin torque switching in nanoscale magnetic tunnel junctions }



Magnetic random access memory based on spin transfer torque effect in nanoscale magnetic tunnel junctions (STT-RAM) is emerging as a promising candidate for embedded and stand-alone computer memory. An important performance parameter of STT-RAM is stability of its free magnetic layer against thermal fluctuations. Measurements of the free layer switching probability as a function of sub-critical voltage at zero effective magnetic field (read disturb rate or RDR measurements) have been proposed as a method for quantitative evaluation of the free layer thermal stability at zero voltage. In this presentation, we report RDR measurement as a function of external magnetic field, which provide a test of the RDR method self-consistency and reliability. 

Probability of switching under 100 ns voltage pulses are measured and plotted for the MTJ as a function of pulse amplitudes. It can be shown that logarithm of switching probability depends to the first order on the thermal stability  as Eq 1. in the sub-critical voltage range. From fitting of the logarithmic slope and intersection we get the thermal stability 



RDR measurements are made for a couple of elliptical MTJ nanopillars of different sizes and structures. All nanopillars are deposited by magnetron sputtering in a Singulus TIMARIS system followed by 2 hours annealing at 300c in a 1 Tesla in-plane magnetic field which sets the pinned layer exchange bias direction parallel to the long axis of the nanopillars.
