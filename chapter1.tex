\chapter{Introduction}

  Spin transfer torque is generated by the transfer of angular momentum from spin polarized electrons to a ferromagnet. This spin transfer torque provides an efficient way to manipulate the dynamic motion of the magnetization of a nanomagnet, and can be strong enough to induce magnetization switching and steady-state precession. This field of study draws enormous attention not only because spin transfer torque is essential in understanding fundamental physical phenomena, but also it makes the building block for future applications such as spin torque oscillators, magnetic random access memory.We have developed several new techniques to characterize such dynamics in nanoscale magnetic tunnel junctions.
  
  In chapter 2, we will first discuss necessary background  knowledge in this field. In chapter 3, we develop a new reliable methods to characterize material parameters such as magnetic anisotropy and Gilbert damping using spin-torque ferromagnetic magnetic resonance with field-modulation. By performing micromagnetic simulations, we can determine other spatial-dependent parameters.
  
  In chapter 4, we demonstrate a single-shot electrical technique to capture the magnetic dynamics during the spin torque switching of a magnetic tunnel junction in real time. With improved sensitivity, we can directly observe real-time oscillation before switching.
  
  In chapter 5, we focus on measurement of switching probability of magnetic tunnel junctions by applying electric pulses. We observe anomalous write error rate behavior in our magnetic tunnel junctions samples. Possible origins of this anomalous write error rate has been discussed.
  
  In chapter 6, we have developed the methods of determine the exchange stiffness of STT-MRAM devices with broken symmetry. Both nominal circular and stadium shape devices have been studied in extensive experimental measurement and micromagnetic simulations. Size-dependent exchange stiffness fitting has been throughly discussed.


%%% Local Variables: ***
%%% mode: latex ***
%%% TeX-master: "thesis.tex" ***
%%% End: ***
